\documentclass[]{article}
\usepackage{lmodern}
\usepackage{amssymb,amsmath}
\usepackage{ifxetex,ifluatex}
\usepackage{fixltx2e} % provides \textsubscript
\ifnum 0\ifxetex 1\fi\ifluatex 1\fi=0 % if pdftex
  \usepackage[T1]{fontenc}
  \usepackage[utf8]{inputenc}
\else % if luatex or xelatex
  \ifxetex
    \usepackage{mathspec}
  \else
    \usepackage{fontspec}
  \fi
  \defaultfontfeatures{Ligatures=TeX,Scale=MatchLowercase}
\fi
% use upquote if available, for straight quotes in verbatim environments
\IfFileExists{upquote.sty}{\usepackage{upquote}}{}
% use microtype if available
\IfFileExists{microtype.sty}{%
\usepackage{microtype}
\UseMicrotypeSet[protrusion]{basicmath} % disable protrusion for tt fonts
}{}
\usepackage[margin=1in]{geometry}
\usepackage{hyperref}
\hypersetup{unicode=true,
            pdftitle={ToothGrowthAnalyses},
            pdfauthor={AlekseyVosk},
            pdfborder={0 0 0},
            breaklinks=true}
\urlstyle{same}  % don't use monospace font for urls
\usepackage{color}
\usepackage{fancyvrb}
\newcommand{\VerbBar}{|}
\newcommand{\VERB}{\Verb[commandchars=\\\{\}]}
\DefineVerbatimEnvironment{Highlighting}{Verbatim}{commandchars=\\\{\}}
% Add ',fontsize=\small' for more characters per line
\usepackage{framed}
\definecolor{shadecolor}{RGB}{248,248,248}
\newenvironment{Shaded}{\begin{snugshade}}{\end{snugshade}}
\newcommand{\KeywordTok}[1]{\textcolor[rgb]{0.13,0.29,0.53}{\textbf{#1}}}
\newcommand{\DataTypeTok}[1]{\textcolor[rgb]{0.13,0.29,0.53}{#1}}
\newcommand{\DecValTok}[1]{\textcolor[rgb]{0.00,0.00,0.81}{#1}}
\newcommand{\BaseNTok}[1]{\textcolor[rgb]{0.00,0.00,0.81}{#1}}
\newcommand{\FloatTok}[1]{\textcolor[rgb]{0.00,0.00,0.81}{#1}}
\newcommand{\ConstantTok}[1]{\textcolor[rgb]{0.00,0.00,0.00}{#1}}
\newcommand{\CharTok}[1]{\textcolor[rgb]{0.31,0.60,0.02}{#1}}
\newcommand{\SpecialCharTok}[1]{\textcolor[rgb]{0.00,0.00,0.00}{#1}}
\newcommand{\StringTok}[1]{\textcolor[rgb]{0.31,0.60,0.02}{#1}}
\newcommand{\VerbatimStringTok}[1]{\textcolor[rgb]{0.31,0.60,0.02}{#1}}
\newcommand{\SpecialStringTok}[1]{\textcolor[rgb]{0.31,0.60,0.02}{#1}}
\newcommand{\ImportTok}[1]{#1}
\newcommand{\CommentTok}[1]{\textcolor[rgb]{0.56,0.35,0.01}{\textit{#1}}}
\newcommand{\DocumentationTok}[1]{\textcolor[rgb]{0.56,0.35,0.01}{\textbf{\textit{#1}}}}
\newcommand{\AnnotationTok}[1]{\textcolor[rgb]{0.56,0.35,0.01}{\textbf{\textit{#1}}}}
\newcommand{\CommentVarTok}[1]{\textcolor[rgb]{0.56,0.35,0.01}{\textbf{\textit{#1}}}}
\newcommand{\OtherTok}[1]{\textcolor[rgb]{0.56,0.35,0.01}{#1}}
\newcommand{\FunctionTok}[1]{\textcolor[rgb]{0.00,0.00,0.00}{#1}}
\newcommand{\VariableTok}[1]{\textcolor[rgb]{0.00,0.00,0.00}{#1}}
\newcommand{\ControlFlowTok}[1]{\textcolor[rgb]{0.13,0.29,0.53}{\textbf{#1}}}
\newcommand{\OperatorTok}[1]{\textcolor[rgb]{0.81,0.36,0.00}{\textbf{#1}}}
\newcommand{\BuiltInTok}[1]{#1}
\newcommand{\ExtensionTok}[1]{#1}
\newcommand{\PreprocessorTok}[1]{\textcolor[rgb]{0.56,0.35,0.01}{\textit{#1}}}
\newcommand{\AttributeTok}[1]{\textcolor[rgb]{0.77,0.63,0.00}{#1}}
\newcommand{\RegionMarkerTok}[1]{#1}
\newcommand{\InformationTok}[1]{\textcolor[rgb]{0.56,0.35,0.01}{\textbf{\textit{#1}}}}
\newcommand{\WarningTok}[1]{\textcolor[rgb]{0.56,0.35,0.01}{\textbf{\textit{#1}}}}
\newcommand{\AlertTok}[1]{\textcolor[rgb]{0.94,0.16,0.16}{#1}}
\newcommand{\ErrorTok}[1]{\textcolor[rgb]{0.64,0.00,0.00}{\textbf{#1}}}
\newcommand{\NormalTok}[1]{#1}
\usepackage{graphicx,grffile}
\makeatletter
\def\maxwidth{\ifdim\Gin@nat@width>\linewidth\linewidth\else\Gin@nat@width\fi}
\def\maxheight{\ifdim\Gin@nat@height>\textheight\textheight\else\Gin@nat@height\fi}
\makeatother
% Scale images if necessary, so that they will not overflow the page
% margins by default, and it is still possible to overwrite the defaults
% using explicit options in \includegraphics[width, height, ...]{}
\setkeys{Gin}{width=\maxwidth,height=\maxheight,keepaspectratio}
\IfFileExists{parskip.sty}{%
\usepackage{parskip}
}{% else
\setlength{\parindent}{0pt}
\setlength{\parskip}{6pt plus 2pt minus 1pt}
}
\setlength{\emergencystretch}{3em}  % prevent overfull lines
\providecommand{\tightlist}{%
  \setlength{\itemsep}{0pt}\setlength{\parskip}{0pt}}
\setcounter{secnumdepth}{0}
% Redefines (sub)paragraphs to behave more like sections
\ifx\paragraph\undefined\else
\let\oldparagraph\paragraph
\renewcommand{\paragraph}[1]{\oldparagraph{#1}\mbox{}}
\fi
\ifx\subparagraph\undefined\else
\let\oldsubparagraph\subparagraph
\renewcommand{\subparagraph}[1]{\oldsubparagraph{#1}\mbox{}}
\fi

%%% Use protect on footnotes to avoid problems with footnotes in titles
\let\rmarkdownfootnote\footnote%
\def\footnote{\protect\rmarkdownfootnote}

%%% Change title format to be more compact
\usepackage{titling}

% Create subtitle command for use in maketitle
\newcommand{\subtitle}[1]{
  \posttitle{
    \begin{center}\large#1\end{center}
    }
}

\setlength{\droptitle}{-2em}
  \title{ToothGrowthAnalyses}
  \pretitle{\vspace{\droptitle}\centering\huge}
  \posttitle{\par}
  \author{AlekseyVosk}
  \preauthor{\centering\large\emph}
  \postauthor{\par}
  \predate{\centering\large\emph}
  \postdate{\par}
  \date{3 June 2018}


\begin{document}
\maketitle

For research we will need functions from external packages.

\begin{Shaded}
\begin{Highlighting}[]
\KeywordTok{library}\NormalTok{(tidyverse)}
\end{Highlighting}
\end{Shaded}

\begin{verbatim}
## -- Attaching packages --------------------------------------------------------------------------------------------------- tidyverse 1.2.1 --
\end{verbatim}

\begin{verbatim}
## v ggplot2 2.2.1     v purrr   0.2.4
## v tibble  1.4.2     v dplyr   0.7.5
## v tidyr   0.8.0     v stringr 1.3.1
## v readr   1.1.1     v forcats 0.3.0
\end{verbatim}

\begin{verbatim}
## -- Conflicts ------------------------------------------------------------------------------------------------------ tidyverse_conflicts() --
## x dplyr::filter() masks stats::filter()
## x dplyr::lag()    masks stats::lag()
\end{verbatim}

\subsection{Data and exploratory data
analyses}\label{data-and-exploratory-data-analyses}

Dataset contanes 60 observations on 3 variables:\\
1. len - Tooth length\\
2. supp - Supplement type: \emph{VC} (ascorbic acid) or \emph{OJ}
(orange juice)\\
3. dose - Dose in milligrams/day

Loading data as tibble and looking at data's structure.

\begin{Shaded}
\begin{Highlighting}[]
\NormalTok{dataset <-}\StringTok{ }\KeywordTok{as_tibble}\NormalTok{(ToothGrowth)}
\KeywordTok{str}\NormalTok{(dataset)}
\end{Highlighting}
\end{Shaded}

\begin{verbatim}
## Classes 'tbl_df', 'tbl' and 'data.frame':    60 obs. of  3 variables:
##  $ len : num  4.2 11.5 7.3 5.8 6.4 10 11.2 11.2 5.2 7 ...
##  $ supp: Factor w/ 2 levels "OJ","VC": 2 2 2 2 2 2 2 2 2 2 ...
##  $ dose: num  0.5 0.5 0.5 0.5 0.5 0.5 0.5 0.5 0.5 0.5 ...
\end{verbatim}

Let's compare tooth length in groups with different doses and different
supplement types.

\begin{Shaded}
\begin{Highlighting}[]
\NormalTok{dataset_f <-}\StringTok{ }\KeywordTok{mutate}\NormalTok{(dataset, }\DataTypeTok{dose =} \KeywordTok{as.factor}\NormalTok{(dose))}
\KeywordTok{table}\NormalTok{(dataset_f}\OperatorTok{$}\NormalTok{supp, dataset_f}\OperatorTok{$}\NormalTok{dose)}
\end{Highlighting}
\end{Shaded}

\begin{verbatim}
##     
##      0.5  1  2
##   OJ  10 10 10
##   VC  10 10 10
\end{verbatim}

\begin{Shaded}
\begin{Highlighting}[]
\KeywordTok{group_by}\NormalTok{(dataset_f, dose) }\OperatorTok
\StringTok{    }\KeywordTok{summarise}\NormalTok{(}\DataTypeTok{len =} \KeywordTok{mean}\NormalTok{(len))}
\end{Highlighting}
\end{Shaded}

\begin{verbatim}
## # A tibble: 3 x 2
##   dose    len
##   <fct> <dbl>
## 1 0.5    10.6
## 2 1      19.7
## 3 2      26.1
\end{verbatim}

\begin{Shaded}
\begin{Highlighting}[]
\KeywordTok{group_by}\NormalTok{(dataset_f, supp) }\OperatorTok
\StringTok{    }\KeywordTok{summarise}\NormalTok{(}\DataTypeTok{len =} \KeywordTok{mean}\NormalTok{(len))}
\end{Highlighting}
\end{Shaded}

\begin{verbatim}
## # A tibble: 2 x 2
##   supp    len
##   <fct> <dbl>
## 1 OJ     20.7
## 2 VC     17.0
\end{verbatim}

Tooth length differs between each groups, to visualise it, let's make
boxplots:

\begin{Shaded}
\begin{Highlighting}[]
\NormalTok{g <-}\StringTok{ }\KeywordTok{ggplot}\NormalTok{(}\DataTypeTok{data =}\NormalTok{ dataset_f, }\DataTypeTok{mapping =} \KeywordTok{aes}\NormalTok{(}\DataTypeTok{y =}\NormalTok{ len)) }\OperatorTok{+}
\StringTok{    }\KeywordTok{labs}\NormalTok{(}\DataTypeTok{y =} \StringTok{"Tooth length"}\NormalTok{, }
         \DataTypeTok{title =} \StringTok{"Effect of Vitamin C on Tooth Growth in Guinea Pigs"}\NormalTok{)}
\NormalTok{g1 <-}\StringTok{ }\NormalTok{g }\OperatorTok{+}\StringTok{ }\KeywordTok{geom_boxplot}\NormalTok{(}\DataTypeTok{mapping =} \KeywordTok{aes}\NormalTok{(}\DataTypeTok{x =}\NormalTok{ dose)) }\OperatorTok{+}
\StringTok{    }\KeywordTok{labs}\NormalTok{(}\DataTypeTok{x =} \StringTok{"Dose of supplement , milligrams/day "}\NormalTok{,}
         \DataTypeTok{title =} \StringTok{"Effect of Vitamin C on Tooth Growth in Guinea Pigs, by dose"}\NormalTok{)}
\NormalTok{g2 <-}\StringTok{ }\NormalTok{g }\OperatorTok{+}\StringTok{ }\KeywordTok{geom_boxplot}\NormalTok{(}\DataTypeTok{mapping =} \KeywordTok{aes}\NormalTok{(}\DataTypeTok{x =}\NormalTok{ supp)) }\OperatorTok{+}
\StringTok{    }\KeywordTok{labs}\NormalTok{(}\DataTypeTok{x =} \StringTok{"Supplement type, OJ (orange juice) or VC (ascorbic acid)"}\NormalTok{,}
         \DataTypeTok{title =} \StringTok{"Effect of Vitamin C on Tooth Growth in Guinea Pigs, by supplement"}\NormalTok{)}
\NormalTok{g3 <-}\StringTok{ }\NormalTok{g }\OperatorTok{+}\StringTok{ }\KeywordTok{geom_boxplot}\NormalTok{(}\DataTypeTok{mapping =} \KeywordTok{aes}\NormalTok{(}\DataTypeTok{x =}\NormalTok{ dose)) }\OperatorTok{+}
\StringTok{    }\KeywordTok{facet_wrap}\NormalTok{(}\OperatorTok{~}\StringTok{ }\NormalTok{supp) }\OperatorTok{+}
\StringTok{    }\KeywordTok{labs}\NormalTok{(}\DataTypeTok{x =} \StringTok{"Dose of supplement , milligrams/day "}\NormalTok{,}
         \DataTypeTok{title =} \StringTok{"Effect of Vitamin C on Tooth Growth in Guinea Pigs, by dose and supplement"}\NormalTok{)}
\end{Highlighting}
\end{Shaded}

Exploratory analyses shows us, than the tooth growth in guinea pigs
differs in groups with different doses of vitamin C, but not so much in
groups with different supplement type.

\subsection{Statistical inference}\label{statistical-inference}

\paragraph{For statistical inference first estimate relationships
between tooth length and supplement
types}\label{for-statistical-inference-first-estimate-relationships-between-tooth-length-and-supplement-types}

\textbf{Hypotheses tests} for tooth length and supplement types:\\
H0: Tooth length and supplement types are \textbf{independent}. Tooth
length dictribution \textbf{does not vary} by the supplement types.\\
HA: Tooth length and supplement types are \textbf{dependent}. Tooth
length dictribution \textbf{vary} by the supplement types.

We compare one numerical and one categorical (with 2 levels and 30 cases
in each category) variables, so I'll use Student's t-Test with
significance level 0.05.

\begin{Shaded}
\begin{Highlighting}[]
\NormalTok{test_len_supp <-}\StringTok{ }\KeywordTok{t.test}\NormalTok{(}\KeywordTok{filter}\NormalTok{(dataset, supp }\OperatorTok{==}\StringTok{ "VC"}\NormalTok{)}\OperatorTok{$}\NormalTok{len, }
                        \KeywordTok{filter}\NormalTok{(dataset, supp }\OperatorTok{==}\StringTok{ "OJ"}\NormalTok{)}\OperatorTok{$}\NormalTok{len,}
                        \DataTypeTok{alternative =} \StringTok{"two.sided"}\NormalTok{,}
                        \DataTypeTok{mu =} \DecValTok{0}\NormalTok{,}
                        \DataTypeTok{paired =} \OtherTok{FALSE}\NormalTok{,}
                        \DataTypeTok{conf.level =} \FloatTok{0.95}\NormalTok{)}
\KeywordTok{paste0}\NormalTok{(}\StringTok{"p-value: "}\NormalTok{,test_len_supp}\OperatorTok{$}\NormalTok{p.value)}
\end{Highlighting}
\end{Shaded}

\begin{verbatim}
## [1] "p-value: 0.0606345078809341"
\end{verbatim}

P-value is above significance level, so we fail to regect H0.\\
We can also see 95\% confidence interval:

\begin{Shaded}
\begin{Highlighting}[]
\NormalTok{test_len_supp}\OperatorTok{$}\NormalTok{conf.int}
\end{Highlighting}
\end{Shaded}

\begin{verbatim}
## [1] -7.5710156  0.1710156
## attr(,"conf.level")
## [1] 0.95
\end{verbatim}

Confidence interval includes zero, so we can't rule out possibility of
indifference between 2 groups.

\paragraph{Now let's estimate relationships between tooth length and
dose of vitamin
C}\label{now-lets-estimate-relationships-between-tooth-length-and-dose-of-vitamin-c}

\textbf{Hypotheses tests} for tooth length and dose:\\
H0: Tooth length and vitamin dose are \textbf{independent}. Tooth length
dictribution \textbf{does not vary} by the vitamin dose.\\
HA: Tooth length and vitamin dose are \textbf{dependent}. Tooth length
dictribution \textbf{vary} by the vitamin dose.

We compare one numerical and one categorical (with 3 levels and 20 cases
in each category) variables. The optimal statistical test would be
ANOVA, but as we need to use technique from the class, let's conduct 3
Student's t-Tests (still 2-sided).\\
We will need to correct our significance level for this 3 test's, for
simplicity let's make Bonferroni correction on 0.05:

\begin{Shaded}
\begin{Highlighting}[]
\KeywordTok{paste0}\NormalTok{(}\StringTok{"Bonferroni significance level: "}\NormalTok{, }\FloatTok{0.05}\OperatorTok{/}\DecValTok{3}\NormalTok{)}
\end{Highlighting}
\end{Shaded}

\begin{verbatim}
## [1] "Bonferroni significance level: 0.0166666666666667"
\end{verbatim}

\begin{Shaded}
\begin{Highlighting}[]
\NormalTok{test0.5_}\DecValTok{1}\NormalTok{ <-}\StringTok{ }\KeywordTok{t.test}\NormalTok{(}\KeywordTok{filter}\NormalTok{(dataset, dose }\OperatorTok{==}\StringTok{ }\FloatTok{0.5}\NormalTok{)}\OperatorTok{$}\NormalTok{len,}
                \KeywordTok{filter}\NormalTok{(dataset, dose }\OperatorTok{==}\StringTok{ }\DecValTok{1}\NormalTok{)}\OperatorTok{$}\NormalTok{len)}
\NormalTok{test1_}\DecValTok{2}\NormalTok{ <-}\StringTok{ }\KeywordTok{t.test}\NormalTok{(}\KeywordTok{filter}\NormalTok{(dataset, dose }\OperatorTok{==}\StringTok{ }\DecValTok{1}\NormalTok{)}\OperatorTok{$}\NormalTok{len,}
                \KeywordTok{filter}\NormalTok{(dataset, dose }\OperatorTok{==}\StringTok{ }\DecValTok{2}\NormalTok{)}\OperatorTok{$}\NormalTok{len)}
\NormalTok{test0.5_}\DecValTok{2}\NormalTok{ <-}\StringTok{ }\KeywordTok{t.test}\NormalTok{(}\KeywordTok{filter}\NormalTok{(dataset, dose }\OperatorTok{==}\StringTok{ }\FloatTok{0.5}\NormalTok{)}\OperatorTok{$}\NormalTok{len,}
                \KeywordTok{filter}\NormalTok{(dataset, dose }\OperatorTok{==}\StringTok{ }\DecValTok{2}\NormalTok{)}\OperatorTok{$}\NormalTok{len)}
\KeywordTok{paste0}\NormalTok{(}\StringTok{"0.5/1 p-value: "}\NormalTok{,test0.5_}\DecValTok{1}\OperatorTok{$}\NormalTok{p.value)}
\end{Highlighting}
\end{Shaded}

\begin{verbatim}
## [1] "0.5/1 p-value: 1.26830072017385e-07"
\end{verbatim}

\begin{Shaded}
\begin{Highlighting}[]
\KeywordTok{paste0}\NormalTok{(}\StringTok{"1/2 p-value: "}\NormalTok{,test1_}\DecValTok{2}\OperatorTok{$}\NormalTok{p.value)}
\end{Highlighting}
\end{Shaded}

\begin{verbatim}
## [1] "1/2 p-value: 1.9064295136718e-05"
\end{verbatim}

\begin{Shaded}
\begin{Highlighting}[]
\KeywordTok{paste0}\NormalTok{(}\StringTok{"0.5/2 p-value: "}\NormalTok{,test0.5_}\DecValTok{2}\OperatorTok{$}\NormalTok{p.value)}
\end{Highlighting}
\end{Shaded}

\begin{verbatim}
## [1] "0.5/2 p-value: 4.39752495936323e-14"
\end{verbatim}

All 3 p-values are much lower, then corrected significance level, so we
can reject H0 in favor of HA.

\subsection{Conclusions}\label{conclusions}

Our statistical inference shows that at usual significance level tooth
length and vitamin doses are associated, but tooth length and supplement
types are not.\\
For this conclusions we must assume, that all pigs were independent to
each other, and each pig corresponds to only one observation in the
dataset.


\end{document}
