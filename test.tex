\documentclass[]{article}
\usepackage{lmodern}
\usepackage{amssymb,amsmath}
\usepackage{ifxetex,ifluatex}
\usepackage{fixltx2e} % provides \textsubscript
\ifnum 0\ifxetex 1\fi\ifluatex 1\fi=0 % if pdftex
  \usepackage[T1]{fontenc}
  \usepackage[utf8]{inputenc}
\else % if luatex or xelatex
  \ifxetex
    \usepackage{mathspec}
  \else
    \usepackage{fontspec}
  \fi
  \defaultfontfeatures{Ligatures=TeX,Scale=MatchLowercase}
\fi
% use upquote if available, for straight quotes in verbatim environments
\IfFileExists{upquote.sty}{\usepackage{upquote}}{}
% use microtype if available
\IfFileExists{microtype.sty}{%
\usepackage{microtype}
\UseMicrotypeSet[protrusion]{basicmath} % disable protrusion for tt fonts
}{}
\usepackage[margin=1in]{geometry}
\usepackage{hyperref}
\hypersetup{unicode=true,
            pdftitle={SimulationExercise},
            pdfauthor={AlekseyVosk},
            pdfborder={0 0 0},
            breaklinks=true}
\urlstyle{same}  % don't use monospace font for urls
\usepackage{color}
\usepackage{fancyvrb}
\newcommand{\VerbBar}{|}
\newcommand{\VERB}{\Verb[commandchars=\\\{\}]}
\DefineVerbatimEnvironment{Highlighting}{Verbatim}{commandchars=\\\{\}}
% Add ',fontsize=\small' for more characters per line
\usepackage{framed}
\definecolor{shadecolor}{RGB}{248,248,248}
\newenvironment{Shaded}{\begin{snugshade}}{\end{snugshade}}
\newcommand{\KeywordTok}[1]{\textcolor[rgb]{0.13,0.29,0.53}{\textbf{#1}}}
\newcommand{\DataTypeTok}[1]{\textcolor[rgb]{0.13,0.29,0.53}{#1}}
\newcommand{\DecValTok}[1]{\textcolor[rgb]{0.00,0.00,0.81}{#1}}
\newcommand{\BaseNTok}[1]{\textcolor[rgb]{0.00,0.00,0.81}{#1}}
\newcommand{\FloatTok}[1]{\textcolor[rgb]{0.00,0.00,0.81}{#1}}
\newcommand{\ConstantTok}[1]{\textcolor[rgb]{0.00,0.00,0.00}{#1}}
\newcommand{\CharTok}[1]{\textcolor[rgb]{0.31,0.60,0.02}{#1}}
\newcommand{\SpecialCharTok}[1]{\textcolor[rgb]{0.00,0.00,0.00}{#1}}
\newcommand{\StringTok}[1]{\textcolor[rgb]{0.31,0.60,0.02}{#1}}
\newcommand{\VerbatimStringTok}[1]{\textcolor[rgb]{0.31,0.60,0.02}{#1}}
\newcommand{\SpecialStringTok}[1]{\textcolor[rgb]{0.31,0.60,0.02}{#1}}
\newcommand{\ImportTok}[1]{#1}
\newcommand{\CommentTok}[1]{\textcolor[rgb]{0.56,0.35,0.01}{\textit{#1}}}
\newcommand{\DocumentationTok}[1]{\textcolor[rgb]{0.56,0.35,0.01}{\textbf{\textit{#1}}}}
\newcommand{\AnnotationTok}[1]{\textcolor[rgb]{0.56,0.35,0.01}{\textbf{\textit{#1}}}}
\newcommand{\CommentVarTok}[1]{\textcolor[rgb]{0.56,0.35,0.01}{\textbf{\textit{#1}}}}
\newcommand{\OtherTok}[1]{\textcolor[rgb]{0.56,0.35,0.01}{#1}}
\newcommand{\FunctionTok}[1]{\textcolor[rgb]{0.00,0.00,0.00}{#1}}
\newcommand{\VariableTok}[1]{\textcolor[rgb]{0.00,0.00,0.00}{#1}}
\newcommand{\ControlFlowTok}[1]{\textcolor[rgb]{0.13,0.29,0.53}{\textbf{#1}}}
\newcommand{\OperatorTok}[1]{\textcolor[rgb]{0.81,0.36,0.00}{\textbf{#1}}}
\newcommand{\BuiltInTok}[1]{#1}
\newcommand{\ExtensionTok}[1]{#1}
\newcommand{\PreprocessorTok}[1]{\textcolor[rgb]{0.56,0.35,0.01}{\textit{#1}}}
\newcommand{\AttributeTok}[1]{\textcolor[rgb]{0.77,0.63,0.00}{#1}}
\newcommand{\RegionMarkerTok}[1]{#1}
\newcommand{\InformationTok}[1]{\textcolor[rgb]{0.56,0.35,0.01}{\textbf{\textit{#1}}}}
\newcommand{\WarningTok}[1]{\textcolor[rgb]{0.56,0.35,0.01}{\textbf{\textit{#1}}}}
\newcommand{\AlertTok}[1]{\textcolor[rgb]{0.94,0.16,0.16}{#1}}
\newcommand{\ErrorTok}[1]{\textcolor[rgb]{0.64,0.00,0.00}{\textbf{#1}}}
\newcommand{\NormalTok}[1]{#1}
\usepackage{graphicx,grffile}
\makeatletter
\def\maxwidth{\ifdim\Gin@nat@width>\linewidth\linewidth\else\Gin@nat@width\fi}
\def\maxheight{\ifdim\Gin@nat@height>\textheight\textheight\else\Gin@nat@height\fi}
\makeatother
% Scale images if necessary, so that they will not overflow the page
% margins by default, and it is still possible to overwrite the defaults
% using explicit options in \includegraphics[width, height, ...]{}
\setkeys{Gin}{width=\maxwidth,height=\maxheight,keepaspectratio}
\IfFileExists{parskip.sty}{%
\usepackage{parskip}
}{% else
\setlength{\parindent}{0pt}
\setlength{\parskip}{6pt plus 2pt minus 1pt}
}
\setlength{\emergencystretch}{3em}  % prevent overfull lines
\providecommand{\tightlist}{%
  \setlength{\itemsep}{0pt}\setlength{\parskip}{0pt}}
\setcounter{secnumdepth}{0}
% Redefines (sub)paragraphs to behave more like sections
\ifx\paragraph\undefined\else
\let\oldparagraph\paragraph
\renewcommand{\paragraph}[1]{\oldparagraph{#1}\mbox{}}
\fi
\ifx\subparagraph\undefined\else
\let\oldsubparagraph\subparagraph
\renewcommand{\subparagraph}[1]{\oldsubparagraph{#1}\mbox{}}
\fi

%%% Use protect on footnotes to avoid problems with footnotes in titles
\let\rmarkdownfootnote\footnote%
\def\footnote{\protect\rmarkdownfootnote}

%%% Change title format to be more compact
\usepackage{titling}

% Create subtitle command for use in maketitle
\newcommand{\subtitle}[1]{
  \posttitle{
    \begin{center}\large#1\end{center}
    }
}

\setlength{\droptitle}{-2em}
  \title{SimulationExercise}
  \pretitle{\vspace{\droptitle}\centering\huge}
  \posttitle{\par}
  \author{AlekseyVosk}
  \preauthor{\centering\large\emph}
  \postauthor{\par}
  \predate{\centering\large\emph}
  \postdate{\par}
  \date{17 May 2018}


\begin{document}
\maketitle

Import packages and set seed for simulations.

\begin{Shaded}
\begin{Highlighting}[]
\KeywordTok{library}\NormalTok{(ggplot2)}
\KeywordTok{library}\NormalTok{(tibble)}
\KeywordTok{set.seed}\NormalTok{(}\DecValTok{100}\NormalTok{)}
\end{Highlighting}
\end{Shaded}

\subsection{Exponential Distribution}\label{exponential-distribution}

We create list of values, associated with our exponential distribution.

\begin{Shaded}
\begin{Highlighting}[]
\NormalTok{exp_distr =}\StringTok{ }\KeywordTok{list}\NormalTok{(}
  \DataTypeTok{lambda =} \FloatTok{0.2}\NormalTok{,}
  \DataTypeTok{mean =} \DecValTok{1}\OperatorTok{/}\FloatTok{0.2}\NormalTok{,}
  \DataTypeTok{sd =} \DecValTok{1}\OperatorTok{/}\FloatTok{0.2}\NormalTok{,}
  \DataTypeTok{var =}\NormalTok{ (}\DecValTok{1}\OperatorTok{/}\FloatTok{0.2}\NormalTok{)}\OperatorTok{^}\DecValTok{2}\NormalTok{,}
  \DataTypeTok{se_expected =}\NormalTok{ (}\DecValTok{1}\OperatorTok{/}\FloatTok{0.2}\NormalTok{)}\OperatorTok{/}\KeywordTok{sqrt}\NormalTok{(}\DecValTok{40}\NormalTok{)    }\CommentTok{# expected standard error of the sample mean, estimated for sample size 40}
\NormalTok{)}
\KeywordTok{print}\NormalTok{(exp_distr)}
\end{Highlighting}
\end{Shaded}

\begin{verbatim}
## $lambda
## [1] 0.2
## 
## $mean
## [1] 5
## 
## $sd
## [1] 5
## 
## $var
## [1] 25
## 
## $se_expected
## [1] 0.7905694
\end{verbatim}

\subsection{Simulations}\label{simulations}

Let's simulate 1000 random samples (n = 40) from exponential
distribution, evaluate mean and variance of each sample and store
results in tibble \textbf{sim\_df}. Also let's evaluet standard error of
means as sd of sample means.

\begin{Shaded}
\begin{Highlighting}[]
\NormalTok{sim_matrix <-}\StringTok{ }\KeywordTok{matrix}\NormalTok{ (}\KeywordTok{rexp}\NormalTok{(}\DecValTok{1000}\OperatorTok{*}\DecValTok{40}\NormalTok{, }\FloatTok{0.2}\NormalTok{), }\DataTypeTok{ncol =} \DecValTok{40}\NormalTok{)}
\NormalTok{sim_means <-}\StringTok{ }\KeywordTok{apply}\NormalTok{(sim_matrix, }\DecValTok{1}\NormalTok{, mean)    }\CommentTok{# means for each of 1000 simulations}
\NormalTok{sim_var <-}\StringTok{ }\KeywordTok{apply}\NormalTok{(sim_matrix, }\DecValTok{1}\NormalTok{, var)       }\CommentTok{# variances for each of 1000 simulations}
\NormalTok{sim_df <-}\StringTok{ }\KeywordTok{tibble}\NormalTok{(}\DataTypeTok{mean =}\NormalTok{ sim_means, }\DataTypeTok{var =}\NormalTok{ sim_var)}

\NormalTok{sim_se <-}\StringTok{ }\KeywordTok{sd}\NormalTok{(sim_means)                    }\CommentTok{# sample standard error of the mean, based on 1000 simulations}
\NormalTok{sample_mean <-}\StringTok{ }\KeywordTok{mean}\NormalTok{(sim_means)             }\CommentTok{# mean of the sampling mean distribution}
\NormalTok{sample_var <-}\StringTok{ }\KeywordTok{mean}\NormalTok{(sim_var)                }\CommentTok{# mean of the sampling variance distribution}
\end{Highlighting}
\end{Shaded}

Additionally, for comparison, let's make 1000 random values from
exponential distribution, and normal distribution with mean and variance
corresponding to expected parameters of sampling distribution of mean.

\begin{Shaded}
\begin{Highlighting}[]
\NormalTok{random_exp <-}\StringTok{ }\KeywordTok{tibble}\NormalTok{(}\DataTypeTok{random1000 =} \KeywordTok{rexp}\NormalTok{(}\DecValTok{1000}\NormalTok{))}
\NormalTok{normal_scale <-}\StringTok{ }\KeywordTok{seq}\NormalTok{(}\DecValTok{2}\NormalTok{, }\DecValTok{8}\NormalTok{, }\FloatTok{0.05}\NormalTok{)}
\NormalTok{normal_value <-}\StringTok{ }\KeywordTok{sapply}\NormalTok{(normal_scale, dnorm, exp_distr}\OperatorTok{$}\NormalTok{mean, exp_distr}\OperatorTok{$}\NormalTok{se_expected)}
\NormalTok{normal_df <-}\StringTok{ }\KeywordTok{tibble}\NormalTok{(}\DataTypeTok{scale =}\NormalTok{ normal_scale, }\DataTypeTok{value =}\NormalTok{ normal_value)}
\end{Highlighting}
\end{Shaded}

\subsection{Exploration}\label{exploration}

\subsubsection{Comparison of sample mean and theoretical mean of
distribution}\label{comparison-of-sample-mean-and-theoretical-mean-of-distribution}

First we evaluate relative difference between this parameters.

\begin{verbatim}
## [1] "theoretical mean" "5"
\end{verbatim}

\begin{verbatim}
## [1] "sample mean"     "4.9997019268744"
\end{verbatim}

\begin{verbatim}
## [1] "relative difference"  "5.96146251199414e-05"
\end{verbatim}

We also can compare standart error of the mean of simulated sampling
distribution and expected value of it.

\begin{verbatim}
## [1] "theoretical standart error" "0.790569415042095"
\end{verbatim}

\begin{verbatim}
## [1] "sample standart error" "0.795946089646085"
\end{verbatim}

\begin{verbatim}
## [1] "relative difference"  "-0.00680101519447738"
\end{verbatim}

We can see that both sample mean and standart error are very close to
there theoretical values.

To illustrate samplin distribution, let's plot it (usind smoothed
version of the histogram) and compare it to the corresponding normal
distribution.

\begin{Shaded}
\begin{Highlighting}[]
\NormalTok{g1 <-}\StringTok{ }\KeywordTok{ggplot}\NormalTok{() }\OperatorTok{+}\StringTok{ }
\StringTok{  }\KeywordTok{geom_density}\NormalTok{(}\DataTypeTok{data =}\NormalTok{ sim_df, }\DataTypeTok{mapping =} \KeywordTok{aes}\NormalTok{(}\DataTypeTok{x =}\NormalTok{ mean), }\DataTypeTok{colour =} \StringTok{"red"}\NormalTok{, }\DataTypeTok{fill =} \StringTok{"red"}\NormalTok{, }\DataTypeTok{alpha =} \FloatTok{0.2}\NormalTok{, }\DataTypeTok{size =} \DecValTok{1}\NormalTok{) }\OperatorTok{+}
\StringTok{  }\KeywordTok{geom_line}\NormalTok{(}\DataTypeTok{data =}\NormalTok{ normal_df, }\DataTypeTok{mapping =} \KeywordTok{aes}\NormalTok{(}\DataTypeTok{x =}\NormalTok{ scale, }\DataTypeTok{y =}\NormalTok{ value), }\DataTypeTok{colour =} \StringTok{"black"}\NormalTok{, }\DataTypeTok{linetype =} \DecValTok{5}\NormalTok{, }\DataTypeTok{size =} \DecValTok{1}\NormalTok{) }\OperatorTok{+}
\StringTok{  }\KeywordTok{geom_vline}\NormalTok{(}\KeywordTok{aes}\NormalTok{(}\DataTypeTok{xintercept =}\NormalTok{ exp_distr}\OperatorTok{$}\NormalTok{mean), }\DataTypeTok{linetype =} \DecValTok{2}\NormalTok{, }\DataTypeTok{size =} \DecValTok{1}\NormalTok{) }\OperatorTok{+}
\StringTok{  }\KeywordTok{geom_vline}\NormalTok{(}\KeywordTok{aes}\NormalTok{(}\DataTypeTok{xintercept =}\NormalTok{ sample_mean), }\DataTypeTok{linetype =} \DecValTok{1}\NormalTok{, }\DataTypeTok{size =} \FloatTok{0.5}\NormalTok{,}\DataTypeTok{colour =} \StringTok{"red"}\NormalTok{) }\OperatorTok{+}
\StringTok{  }\KeywordTok{labs}\NormalTok{(}\DataTypeTok{x =} \StringTok{"Sample mean"}\NormalTok{, }\DataTypeTok{title =} \StringTok{"Comparison simulated sampling mean and corresponding normal distributions"}\NormalTok{)}
\KeywordTok{print}\NormalTok{(g1)}
\end{Highlighting}
\end{Shaded}

\includegraphics{test_files/figure-latex/unnamed-chunk-7-1.pdf}

Soild red line represents samplin distribution of mean, whereas black
dashed line represents normal distribution. This lines are very close
even for sample size 40, and we may assume, that normal distribution is
good approximation for sampling distribution.

\subsubsection{Comparison of sample variance and theoretical variance of
distribution}\label{comparison-of-sample-variance-and-theoretical-variance-of-distribution}

Let's evaluaete relative difference between sample variance and
theoretical variance.

\begin{verbatim}
## [1] "theoretical variance" "25"
\end{verbatim}

\begin{verbatim}
## [1] "sample variance"  "25.3828234274576"
\end{verbatim}

\begin{verbatim}
## [1] "relative difference" "-0.0153129370983031"
\end{verbatim}

There is less then 2\% difference between sample and theoretical
variance, so they are close.

Now let's plot sample variance distribution.

\begin{Shaded}
\begin{Highlighting}[]
\NormalTok{g2 <-}\StringTok{ }\KeywordTok{ggplot}\NormalTok{() }\OperatorTok{+}\StringTok{ }
\StringTok{  }\KeywordTok{geom_density}\NormalTok{(}\DataTypeTok{data =}\NormalTok{ sim_df, }\DataTypeTok{mapping =} \KeywordTok{aes}\NormalTok{(}\DataTypeTok{x =}\NormalTok{ var), }\DataTypeTok{colour =} \StringTok{"blue"}\NormalTok{, }\DataTypeTok{fill =} \StringTok{"blue"}\NormalTok{, }\DataTypeTok{alpha =} \FloatTok{0.2}\NormalTok{, }\DataTypeTok{size =} \DecValTok{1}\NormalTok{) }\OperatorTok{+}
\StringTok{  }\KeywordTok{geom_vline}\NormalTok{(}\KeywordTok{aes}\NormalTok{(}\DataTypeTok{xintercept =}\NormalTok{ sample_var), }\DataTypeTok{linetype =} \DecValTok{1}\NormalTok{, }\DataTypeTok{size =} \FloatTok{0.5}\NormalTok{,}\DataTypeTok{colour =} \StringTok{"blue"}\NormalTok{) }\OperatorTok{+}
\StringTok{  }\KeywordTok{labs}\NormalTok{(}\DataTypeTok{x =} \StringTok{"Sample variance"}\NormalTok{, }\DataTypeTok{title =} \StringTok{"Distribution of sample variance"}\NormalTok{)}
\KeywordTok{print}\NormalTok{(g2)}
\end{Highlighting}
\end{Shaded}

\includegraphics{test_files/figure-latex/unnamed-chunk-9-1.pdf}

Sample variance distribution still looks bell-shaped, however it's
right-skewed.

\subsubsection{Comparison of sample mean and big sample
distributions}\label{comparison-of-sample-mean-and-big-sample-distributions}

Now let's compare our sample mean distribution and distribution of 1000
randomly generated values from exponential distribution.

\begin{Shaded}
\begin{Highlighting}[]
\NormalTok{g3 <-}\StringTok{ }\KeywordTok{ggplot}\NormalTok{() }\OperatorTok{+}\StringTok{ }
\StringTok{  }\KeywordTok{geom_density}\NormalTok{(}\DataTypeTok{data =}\NormalTok{ sim_df, }\DataTypeTok{mapping =} \KeywordTok{aes}\NormalTok{(}\DataTypeTok{x =}\NormalTok{ mean), }\DataTypeTok{colour =} \StringTok{"red"}\NormalTok{, }\DataTypeTok{fill =} \StringTok{"red"}\NormalTok{, }\DataTypeTok{alpha =} \FloatTok{0.2}\NormalTok{, }\DataTypeTok{size =} \DecValTok{1}\NormalTok{) }\OperatorTok{+}
\StringTok{  }\KeywordTok{geom_density}\NormalTok{(}\DataTypeTok{data =}\NormalTok{ random_exp, }\DataTypeTok{mapping =} \KeywordTok{aes}\NormalTok{(}\DataTypeTok{x =}\NormalTok{ random1000), }\DataTypeTok{fill =} \StringTok{"yellow"}\NormalTok{, }\DataTypeTok{alpha =} \FloatTok{0.1}\NormalTok{, }\DataTypeTok{size =} \DecValTok{1}\NormalTok{) }\OperatorTok{+}
\StringTok{  }\KeywordTok{labs}\NormalTok{(}\DataTypeTok{x =} \StringTok{"values"}\NormalTok{, }\DataTypeTok{title =} \StringTok{"Comparison of sample mean and big sample distributions"}\NormalTok{)}
\KeywordTok{print}\NormalTok{(g3)}
\end{Highlighting}
\end{Shaded}

\includegraphics{test_files/figure-latex/unnamed-chunk-10-1.pdf}

As we can see, while our sampling distribution (red on the plot) looks
nearly normal, distribution of big sample (n=1000) of random
exponentials (yellow on the plot) is highly skewed and isn't
approximated by normal distribution.

\subsubsection{Conclusions}\label{conclusions}

Distributions of 1000 sample means and variances from exponential
distribution of size 40 looks nearly normal, and there means are close
to the theoretical exponential distribution mean and variance.


\end{document}
